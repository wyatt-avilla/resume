% Original author:
% Trey Hunner (http://www.treyhunner.com/)

\documentclass{resume}

\usepackage[left=0.75in,top=0.5in,right=0.75in,bottom=0.6in]{geometry}
\usepackage{hyperref}
\usepackage{fontawesome5}
\usepackage{graphicx}

\newcommand{\tab}[1]{\hspace{.2667\textwidth}\rlap{#1}}
\newcommand{\itab}[1]{\hspace{0em}\rlap{#1}}
\newcommand{\scaledfaExternalLink}{\raisebox{0.1\height}{\scalebox{0.7}{\faExternalLink*}}}

\name{Wyatt Avilla}
\address{(408)506-2189}
\address{\href{mailto:wyattmurphy1@gmail.com}{wyattmurphy1@gmail.com}}
\address{\href{https://github.com/wyatt-avilla}{github.com/wyatt-avilla
\scaledfaExternalLink}}

\begin{document}

\begin{rSection}{Education}

  \textbf{San José State University} \hfill{} \textit{Aug 2025 - May 2027} \\
  \textit{Major:} M.S Software Engineering, specializing in
  networking software \hfill{} \\

  \textbf{University of California, Santa Cruz} \hfill{} \textit{Sept
  2021 - March 2025} \\
  \textit{Major:} B.S. Cognitive Science, specializing in AI \& HCI
  \hfill{} \textbf{CGPA:\@ 3.9}
  \href{https://github.com/wyatt-avilla/resume/blob/main/assets/ucsc_official_transcript.pdf}{Transcript
  \scaledfaExternalLink} \\
  \textit{Minor:} Computer Science

  \begin{tabular}{ @{} >{\bfseries}l @{\hspace{6ex}} p{0.7\textwidth} }
    Relevant Courses & Data Structures \& Algorithms, Object Oriented
    Programming, \newline{}
    Parallel Programming, Computer System Design, Artificial
    Intelligence                   \\
  \end{tabular}

\end{rSection}

\begin{rSection}{Technical Strengths}

  \begin{tabular}{ @{} >{\bfseries}l @{\hspace{6ex}} l }
    Programming Languages & C/C++, Python (Pandas, NumPy), Rust,
    Shell, Nix, Lua   \\
    Software \& Tools     & Git, Linux, GitHub Actions, Docker,
    NixOS, WebAssembly \\
  \end{tabular}

\end{rSection}

\begin{rSection}{Work Experience}

  \begin{rSubsection}{Python Developer Intern}{Sept 2024 - Dec 2024}{Lillup}{}
  \item{} Developed a custom parser for an internal markup language,
    emphasizing type safety and maintainability through static typing with Mypy
  \item{} Architected and implemented a fully typed API using the
    LangChain framework, incorporating comprehensive testing and
    CI/CD pipelines through GitHub Actions
  \item{} Demonstrated project leadership through GitHub ecosystem
    utilization (Issues, Wiki, Actions), coordinating technical
    initiatives and maintaining high code quality standards
  \end{rSubsection}

  \begin{rSubsection}{Data Structures \& Algorithms Tutor}{July 2024
    - Sept 2024, Jan 2025 - June 2025}{University of California, Santa Cruz}{}
  \item{} Led group sessions and provided one-on-one assistance to
    students in data structures and algorithms concepts
  \item{} Developed and curated supplemental learning materials,
    including exam preparation resources and practice problems
  \end{rSubsection}

\end{rSection}

\begin{rSection}{Projects}

\item{} {\bf{}
    \href{https://github.com/search?q=repo\%3Adoldecomp\%2Fmelee++author\%3Awyatt-avilla&type=pullrequests&ref=advsearch}{PowerPC
  Assembly Reverse Engineering \scaledfaExternalLink}} {\hfill{} March 2024} \\
  Contributed to an open-source project to reverse-engineer
  \textit{Super Smash Bros. Melee}, working to translate PowerPC
  assembly into C. Collaborated with a team of developers to improve
  the codebase's accuracy, functionality, and documentation.

\item{} {\bf{} \href{https://github.com/wyatt-avilla/sunbird}{Neural
  Network Decompiler Pipeline \scaledfaExternalLink}} {\hfill{} Sept 2024} \\
  Developed a pipeline to train neural networks for assembly
  decompilation by processing C code through multiple compilers and
  optimization levels. Utilized PyTorch for model training and
  Tree-sitter for efficient tokenization and vectorization.

\item{} {\bf{} \href{https://github.com/wyatt-avilla/feframe}{Rust-Based
  Website \scaledfaExternalLink}} {\hfill{} June 2024} \\
  Built a Rust-based website with WebAssembly, dynamically generating
  HTML/CSS using procedural macros, hosted on shuttle.rs.

\end{rSection}

\end{document}
